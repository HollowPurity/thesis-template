\documentclass[conference,compsoc,final,a4paper]{IEEEtran}
\usepackage[utf8]{inputenx}

%% Bitte legen Sie hier den Titel und den Autor der Arbeit fest
\newcommand{\autoren}[0]{Hammond, Gregory}
\newcommand{\dokumententitel}[0]{Modularisierung von Jenkins-Pipelines: Konzepte und Strategien für Microservice-Projekte im privaten Umfeld}

% Hie muss normalerweise nichts angepasst werden
\usepackage[pdftex]{graphicx}
\graphicspath{{img/}}
\DeclareGraphicsExtensions{.pdf,.jpeg,.jpg,.png}
\usepackage[cmex10]{amsmath}
\usepackage{algorithmic}
\usepackage{array}
\usepackage{dblfloatfix}
\usepackage{url}
\usepackage[autostyle=true,german=quotes]{csquotes}
\usepackage[backend=biber,
            sorting=none,   % Keine Sortierung
            doi=true,       % DOI anzeigen
            isbn=false,     % ISBN nicht anzeigen
            url=true,       % URLs anzeigen
            maxnames=6,     % Ab 6 Autoren et al. verwenden
            minnames=1,     % und nur den ersten Autor angeben
            style=ieee,]{biblatex}
\usepackage{booktabs}
\usepackage{xcolor}
\usepackage{listings}             % Source Code listings
\usepackage[printonlyused]{acronym}
\usepackage{fancyvrb}
\usepackage{tocloft} % Schönere Inhaltsverzeichnisse

% Farben definieren
\definecolor{linkblue}{RGB}{0, 0, 100}
\definecolor{linkblack}{RGB}{0, 0, 0}
\definecolor{darkgreen}{RGB}{14, 144, 102}
\definecolor{darkblue}{RGB}{0,0,168}
\definecolor{darkred}{RGB}{128,0,0}
\definecolor{comment}{RGB}{63, 127, 95}
\definecolor{javadoccomment}{RGB}{63, 95, 191}
\definecolor{keyword}{RGB}{108, 0, 67}
\definecolor{type}{RGB}{0, 0, 0}
\definecolor{method}{RGB}{0, 0, 0}
\definecolor{variable}{RGB}{0, 0, 0}
\definecolor{literal}{RGB}{31,0, 255}
\definecolor{operator}{RGB}{0, 0, 0}

\usepackage[ngerman]{betababel}

\DefineBibliographyStrings{ngerman}{
    andothers = {{et al\adddot}},  % Immer et al. sagen, auch bei Deutsch als Sprache
}
\usepackage[
      unicode=true,
      hypertexnames=false,
      colorlinks=true,
      colorlinks=false,
      linkcolor=darkblue,
      citecolor=darkblue,
      urlcolor=darkblue,
      pdftex
   ]{hyperref}
%	 \PrerenderUnicode{ü}


% Einstellungen für Quelltexte
\lstset{
    xleftmargin=0.1cm,
    basicstyle=\scriptsize\ttfamily,
    keywordstyle=\color{keyword},
    identifierstyle=\color{variable},
    commentstyle=\color{comment},
    stringstyle=\color{literal},
    tabsize=2,
    lineskip={2pt},
    columns=flexible,
    inputencoding=utf8,
    captionpos=b,
    breakautoindent=true,
    breakindent=2em,
    breaklines=true,
    prebreak=,
    postbreak=,
    numbers=none,
    numberstyle=\tiny,
    showspaces=false,      % Keine Leerzeichensymbole
    showtabs=false,        % Keine Tabsymbole
    showstringspaces=false,% Leerzeichen in Strings
    morecomment=[s][\color{javadoccomment}]{/**}{*/},
    literate={Ö}{{\"O}}1 {Ä}{{\"A}}1 {Ü}{{\"U}}1 {ß}{{\ss}}2 {ü}{{\"u}}1 {ä}{{\"a}}1 {ö}{{\"o}}1
}

\hypersetup{
    pdftitle={\dokumententitel},
    pdfauthor={\autoren},
    pdfdisplaydoctitle=true,
    hidelinks
}

% Makros für typographisch korrekte Abkürzungen
\newcommand{\zb}[0]{z.\,B.}
\newcommand{\dahe}[0]{d.\,h.}
\newcommand{\ua}[0]{u.\,a.}

% Wo liegt Sourcecode?
\newcommand{\srcloc}{src/}

% Literatur einbinden
\addbibresource{literatur.bib} % Weitere Einstellungen aus einer anderen Datei lesen

\begin{document}

% Titel des Dokuments
\title{\dokumententitel}

% Namen der Autoren
\author{
  \IEEEauthorblockN{\autoren}
  \IEEEauthorblockA{
    Technische Hochschule Mannheim\\
    Fakultät für Informatik\\
    Paul-Wittsack-Str. 10,
    68163 Mannheim
    }
}

% Titel erzeugen
\maketitle
\thispagestyle{plain}
\pagestyle{plain}

% Eigentliches Dokument beginnt hier
% ----------------------------------------------------------------------------------------------------------

% Kurze Zusammenfassung des Dokuments
\begin{abstract}
An dieser Stelle steht eine kurze Zusammenfassung des Inhaltes des Dokuments. Der Abstrakt ist vollkommen eigenständig und hat weder Querverweise zu anderen Teilen dieser Arbeit noch Referenzen zu Quellen.

Schreiben Sie die Zusammenfassung im sogenannten faktenzentrierten Stil, d.\,h. beschreiben Sie nicht das Dokument, sondern die Fakten und Informationen, die das Dokument liefert. Zum Beispiel schreiben Sie \textit{nicht} \enquote{Dieses Dokument stellt die besondere Bedeutung des Flux-Kompensators für Zeitreisen im Film \enquote{Zurück in die Zukunft} dar.}, sondern schreiben Sie \enquote{Ohne den Flux-Kompensator wären die Zeitreisen im Film \enquote{Zurück in die Zukunft} nicht möglich.}

Den Abstract schreibt man als letztes.
\end{abstract}

% Inhaltsverzeichnis erzeugen
{\small\tableofcontents}

% --------------------------------------------------------------------
\section{Einleitung}
\subsection{Motivation und Problemstellung}
\subsection{Zielsetzung und Fragestellung}
\subsection{Aufbau der Arbeit}

\section{Grundlagen}
\subsection{Continuous Integration und Continuous Deployment}
\subsection{Überblick über Jenkins und seine Pipeline-Architektur}
\subsection{Declarative vs. Scripted Pipelines}
\subsection{Modularisierung und Wiederverwendbarkeit in CI/CD}

\section{Analyse}
\subsection{Kriterien für modulare Pipelines im Microservice-Kontext}
\subsection{Vergleich verschiedener Modularisierungsansätze in Jenkins}
\subsection{Sicherheits- und Wartbarkeitsaspekte}

\section{Konzeptioneller Vorschlag}
\subsection{Pipeline-Design für ein hypothetisches D\&D-Charaktertool}
\subsection{Stufenstruktur (Build, Test, Deployment)}
\subsection{Reflexion von Herausforderungen und Grenzen}

\section{Diskussion}
\subsection{Übertragbarkeit auf andere Projekte}
\subsection{Abgrenzung zu professionellen CI/CD-Setups}

\section{Fazit}
\subsection{Zusammenfassung zentraler Erkenntnisse}
\subsection{Beantwortung der Fragestellung}
\subsection{Ausblick}
% --------------------------------------------------------------------
\section*{Abkürzungen}
\addcontentsline{toc}{section}{Abkürzungen}

% Die längste Abkürzung wird in die eckigen Klammern
% bei \begin{acronym} geschrieben, um einen hässlichen
% Umbruch zu verhindern
% Sie müssen die Abkürzungen selbst alphabetisch sortieren!
\begin{acronym}[IEEE]
\acro{A2A}{Application-to-Application}
\acro{ABK}{Abkürzung}
\acro{ACL}{Acess Control List}
\acro{ACM}{Association of Computing Machinery}
\acro{AES}{Advanced Encryption Standard}
\acro{IEEE}{Institute of Electrical and Electronics Engineers}
\acro{ISO}{International Organization for Standardization}
\acro{PDF}{Portable Document Format}
\end{acronym}

% Literaturverzeichnis
\addcontentsline{toc}{section}{Literatur}
\printbibliography

\end{document}